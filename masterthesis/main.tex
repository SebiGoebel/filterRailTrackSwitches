\documentclass[a4paper,12pt]{article}

% Pakete
\usepackage[utf8]{inputenc} % Umlaute und Akzente
\usepackage[T1]{fontenc} % Zeichensatzkodierung
\usepackage{lmodern} % Moderne Schriftart
\usepackage{geometry} % Seitenränder anpassen
\geometry{a4paper, margin=1in}
\usepackage{amsmath, amssymb} % Mathematische Symbole
\usepackage{graphicx} % Bilder einfügen
\usepackage{hyperref} % Hyperlinks
\usepackage{csquotes} % Anführungszeichen korrekt setzen
\usepackage{babel} % Sprachunterstützung
\usepackage{biblatex} % Literaturverzeichnis
\addbibresource{literatur.bib} % Literaturdatei

% Titelinformationen
\title{Der Titel deiner Arbeit}
\author{Dein Name}
\date{\today} % Setzt das aktuelle Datum

\begin{document}

% Titel
\maketitle

% Abstract
\begin{abstract}
    hdfjsahfj asyd asd
    Hallo halofhdsahfio Dies ist eine kurze Zusammenfassung deiner Arbeit. Hier gibst du einen Überblick über die wichtigsten Punkte, Methoden und Ergebnisse.
\end{abstract}

% Einleitung
\section{Einleitung}
Hier beginnt die Einleitung deiner Arbeit. Du erklärst das Thema und führst in die Fragestellung ein. 

% Theorie oder Hintergrund
\section{Hintergrund}
Hier beschreibst du den theoretischen Hintergrund oder relevante Vorarbeiten zu deinem Thema. 

% Methoden
\section{Methoden}
In diesem Abschnitt erklärst du die verwendeten Methoden und Vorgehensweisen.

% Ergebnisse
\section{Ergebnisse}
Hier präsentierst du die Ergebnisse deiner Arbeit. Du kannst auch Tabellen und Abbildungen einfügen. Zum Beispiel:

% Diskussion
\section{Diskussion}
Diskutiere die Bedeutung der Ergebnisse, vergleiche sie mit der Literatur und stelle Bezüge her.

% Fazit
\section{Fazit}
Schließe die Arbeit mit einem Fazit und einem Ausblick auf zukünftige Arbeiten.

% Literaturverzeichnis
\printbibliography

\end{document}
